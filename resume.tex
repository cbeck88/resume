%______________________________________________________________________________________________________________________
% @brief    LaTeX2e Resume for Kamil K Wojcicki
\documentclass[margin,line]{resume}


%______________________________________________________________________________________________________________________
\begin{document}
\name{\Large Chris Beck}
\begin{resume}

    %__________________________________________________________________________________________________________________
    % Contact Information
    \section{\mysidestyle Contact\\Information}

                                                   %      \hfill office: +1 (609) 258 1641          \vspace{0mm}\\\vspace{0mm}%
                                                          \hfill mobile: (upon request)          \vspace{0mm}\\\vspace{0mm}%
                                                          \hfill e-mail: beck.ct@gmail.com  \vspace{0mm}\\\vspace{-4.5mm}%


    %__________________________________________________________________________________________________________________
    % Research Interests
    % \section{\mysidestyle Research\\Interests}

    % Computational Complexity, Algorithms, Combinatorics \\ 


    %__________________________________________________________________________________________________________________
    % Education
    \section{\mysidestyle Education}

    \textbf{Princeton University}, Princeton, New Jersey, USA \vspace{2mm}\\\vspace{1mm}%
    \textsl{Doctor of Philosophy: Computer Science} \hfill \textbf{ 2009 -- 2014 }\vspace{-3mm}\\\vspace{-1mm}%
    \begin{list2}
        \item Advisors:  Professor Sanjeev Arora and Professor Russell Impagliazzo
        \item GPA: 3.940/4.0
    \end{list2}\vspace{-1.5mm}
%    \textsl{BInfTech, BEng (Hons)} \hfill \textbf{February 2000 -- April 2005}\vspace{-3mm}\\\vspace{-1mm}%

    \textbf{California Institute of Technology}, Pasadena, California, USA \vspace{2mm}\\\vspace{1mm}%
    \textsl{Bachelors of Science with Honor: Mathematics} \hfill \textbf{ 2005 -- 2009}\\
    \textsl{Bachelors of Science with Honor: Computer Science} \hfill \textbf{ 2005 -- 2009}\vspace{-3mm}\\%\vspace{-1mm}
    \begin{list2}
	\item GPA: 3.7/4.0
    \end{list2}

    \textbf{Selected course titles:} Algorithms and Data structures, Information security, Machine Learning \\
    Algebra, Geometry, Topology, Analysis, Probability, Combinatorics, Algebraic Geometry \\

    %__________________________________________________________________________________________________________________
    % Publications
    \section{\mysidestyle Selected Publications}

    P. Beame, C. Beck, R. Impagliazzo. Time-Space Tradeoffs in Resolution: Superpolynomial Lower Bounds for Superlinear Space. \textsl{Proceedings of the 44th Annual ACM Symposium on Theory of Computing } (STOC 2012). {\bf Also in special issue of SIAM Journal on Computing 2016 45:4, 1612-1645 .}

\vspace{-2mm}
    C. Beck, R. Impagliazzo, S. Lovett. Large Deviation Bounds for Decision Trees and Sampling Lower Bounds for AC0-circuits. \textsl{Proceedings of the 53rd Annual IEEE Symposium on Foundations of Computer Science} (FOCS 2012). 

\vspace{-2mm}
    C. Beck, J. Nordstr{\"o}m, B. Tang. Some Tradeoffs in Polynomial Calculus. 
    \textsl{Proceedings of the 45th Annual ACM Symposium on Theory of Computing } (STOC 2013).
    

\vspace{-2mm}
    C. Beck, R. Impagliazzo. Strong ETH Holds for Regular Resolution.
    \textsl{Proceedings of the 45th Annual ACM Symposium on Theory of Computing } (STOC 2013).

    %__________________________________________________________________________________________________________________
    % Honours and Awards
    \section{\mysidestyle Honors and\\Awards} 

    Wu Prize for Excellence, 2013 \vspace{1mm}\\
    {\bf Simons Award for Graduate Students in Theoretical Computer Science}, 2012-2014		    \vspace{1mm}\\
    NSF GRFP Honorable Mention, 2009								    \vspace{1mm}\\%
    The G. Wallace Ruckert '30 Fellowship, 2009 						    \vspace{1mm}\\%
    SURF Fellow, 2006 and 2008									    \vspace{1mm}\\%
    
    %__________________________________________________________________________________________________________________
    % Professional Experience
    \section{\mysidestyle Professional\\Experience}

    \textbf{Institute for Advanced Study}, Princeton, New Jersey, USA \vspace{2mm}\\\vspace{1mm}%
    \textsl{Postdoctoral Scholar} \hfill \textbf{Sept 2014 -- August 2016}\\
    Research in computational complexity theory, especially time space tradeoffs and pseudorandomness.

    \textbf{Princeton University}, Princeton, New Jersey, USA \vspace{2mm}\\\vspace{1mm}%
%    \textsl{Research Assistant} \hfill \textbf{June 2011 -- Sept 2011}\\
%    Investigated time space tradeoffs for proof systems and other models, concluding our work on [BBI'12]. Made a conjecture implying the main result of [BIL'12] which we later proved.
    \textsl{Teaching Assistant} \hfill \textbf{Sept 2011 -- May 2012}\\
    Teaching assistant for undergraduate courses in theory of computation, computational geometry.
%    \textsl{Research Assistant} \hfill \textbf{June 2010 -- Sept 2010}\\
%    Investigated the classical complexity of several group-theoretic problems conjectured to have quantum speedups.
%    In an unpublished manuscript, gave evidence that this is unlikely to be the case.

    \textbf{California Institute of Technology}, Pasadena, California, USA \vspace{2mm}\\\vspace{1mm}%
    \textsl{Teaching Assistant} \hfill \textbf{Mar 2009 -- June 2009}\\
    Teaching assistant for undergraduate course in approximation algorithms.

%\newpage

    \textsl{SURF Research Fellow} \hfill \textbf{June 2008 -- September 2008}\\
    Investigated Polynomial Calculus proofs of Graph Nonisomorphism, and related issues in algebraic graph theory.
    Mentored by Professor Richard Wilson.
       

\newpage

    %__________________________________________________________________________________________________________________
    % Computer Skills
    \section{\mysidestyle Programming} 

    C, C++, Python, Lua, Matlab, Bash scripting, Git

    \section{\mysidestyle Open Source\\Contributions}

    \textbf{visit\_struct} (Lead developer) \hfill https://github.com/cbeck88/visit\_struct \\
    A tiny library that provides for struct-field reflection in C++11. It is portable to many versions of gcc, clang, and msvc,
    and I'm told in github issue comments that it has been used in production.

    \textbf{strict\_variant} (Lead developer) \hfill https://github.com/cbeck88/strict\_variant \\
    A simple and efficient type-safe union for C++11.

    \textbf{spirit\_po} (Lead developer) \hfill https://github.com/cbeck88/spirit\_po \\
    A library that parses the gettext po format, and reproduces parts of the interface of libintl.
    It can be used in C++ projects that use the GNU gettext system for internationalization and localization of software.
    spirit\_po is written using the boost::spirit high-level parser framework, it is in total about 900 lines of code.
    spirit\_po has been used for about a year by the Battle for Wesnoth project.

    \textbf{CEGUI} \hfill https://cegui.org.uk  \\
    CEGUI is a GUI framework written in C++ and distributed under a permissive license. It is highly performant,
    flexible, and configurable, with minimal dependencies. The project is at least ten years old and has been used 
    in hundreds of projects, including AAA game titles like Torchlight. \\
    % \begin{itemize}
    % \item
    \textsl{Porting to WebGL} \hfill https://github.com/cbeck88/cegui-emscripten \\
    I ported the CEGUI samples framework to javascript using the emscripten cross compiler.
    This means the user can view the samples in their webbrowser, rendered using WebGL. This uncovered some
    bugs in the CEGUI OpenGL renderer, and I submitted a patch upstream to fix these problems. \\
    % \item
    % \textsl{Menubar behavior} CEGUI contains built-in functionality for menus, menubars, context menus and other staples
    % of UI. However, they had the problem that sometimes they would go off the screen unnecessarily and the user wouldn't
    % be able to see them or use them. Modern users expect that menus will intelligently position themselves to avoid being
    % clipped. I submitted a patch so that CEGUI would do this also, and negotiated it's merger into the default branch. \\
    % \end{itemize}

    \textbf{wesnoth} \hfill https://github.com/wesnoth/wesnoth \\
    The Battle for Wesnoth is a turn-based strategy game developed as a community project since about 2004.
    I made patches to this project in the years 2014-2015, and I have commit access to it. \\
    % \begin{itemize}
    %\item
    \textsl{WML API Unit tests} \\ I created a unit testing framework for the WML API. WML is a data language that is used
    to make wesnoth scenarios. Previously there were no automated tests of this API. I created a framework for unattended
    test scenarios, and a framework to run them as part of continuous integration. The intial tests I created have been
    extended by many developers and the framework is still used today.
    These tests also permitted aggressive refactoring of the project which resulted
    in about 10,000 lines of code being removed. \\
    % \item
    \textsl{xBRZ scaling engine} \\ Wesnoth graphics are based on software rendered sprites, drawn in a pixel art style. In several patches,
    I fixed some bugs in scaling algorithms that had been used, and merged a new patch which allowed the use of a modern scaling algorithm
    called xBRZ. xBRZ attempts to scale pixel art in a way that preserves edges and angles. \\
    % \end{itemize}


    \section{\mysidestyle Invited Talks}

    \begin{list2}
	\item {\bf John Templeton Foundation Workshop}: ``Limits of Theorem Proving''. Rome, Sept 2012.
	\item {\bf China Theory Week} Aarhus, August 2012.
	\item {\bf Symposium on Theory of Computing} New York City, May 2012.
	\item {\bf University of Toronto} Theory Group Seminar. May 2012.
	\item {\bf KTH Royal Institute of Technology} Theory Group Seminar. January 2012.
	\item {\bf Institute for Advanced Study} Computer Science Discrete Math Seminar. December 2011.
	\item {\bf University of Chicago} Theory Group Seminar. November 2011.
	\item {\bf BIRS Workshop}: ``Proof Complexity''. Banff, October 2011.
	\item {\bf University of Chicago} Theory Group Seminar. December 2012.
	\item {\bf Symposium on Foundations of Computer Science} New Brunswick, September 2012.
	\item {\bf Symposium on Theory of Computing} Palo Alto, May 2013.
	\item {\bf University of California, San Diego} Theory Group Seminar, August 2013.
    \item {\bf Insitute for Advanced Study} Computer Science Discrete Math Seminar. March 2015.
    \end{list2}


    %__________________________________________________________________________________________________________________
    % Referees
%    \section{\mysidestyle Referees} 
%    {\sl Available on request.}



%______________________________________________________________________________________________________________________
% \newpage
% \section{\mysidestyle References} 
% \emph{Additional contact information available upon request.}

% \begin{tabular}{@{}p{6cm}p{6cm}}
% \textbf{Professor Sanjeev Arora}        &  \textbf{Professor Russell Impagliazzo}   \\
% Professor of Computer Science           &  Professor of Computer Science            \\
% Princeton University                    &  University of California, San Diego      \\
% Princeton, New Jersey, USA              &  San Diego, California, USA               \\
%phone: \textsl{available on request}    &  phone: \textsl{available on request}     \\
% e-mail: arora@cs.princeton.edu        &  e-mail: russell@cs.ucsd.edu    \\
% \end{tabular}

% \begin{tabular}{@{}p{6cm}p{6cm}}
% \textbf{Professor Shachar Lovett}    &  \textbf{Professor Paul Beame}        \\
% Asst. Professor of Computer Science   &  Professor of Computer Science                 \\
% University of California, San Diego       &  University of Washington       \\
% San Diego, California, USA                           &  Seattle, Washington, USA                \\
%phone: \textsl{available on request}    &  phone: \textsl{available on request}     \\
% e-mail: slovett@math.ias.edu           &  e-mail: beame@cs.washington.edu    \\
% \end{tabular}



%______________________________________________________________________________________________________________________
\end{resume}
\end{document}


%______________________________________________________________________________________________________________________
% EOF

