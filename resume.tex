%______________________________________________________________________________________________________________________
% @brief    LaTeX2e Resume for Kamil K Wojcicki
\documentclass[margin,line]{resume}


%______________________________________________________________________________________________________________________
\begin{document}
\name{\Large Chris Beck}
\begin{resume}

    %__________________________________________________________________________________________________________________
    % Contact Information
    \section{\mysidestyle Contact\\Information}

                                                   %      \hfill office: +1 (609) 258 1641          \vspace{0mm}\\\vspace{0mm}%
                                                          \hfill mobile: (upon request)          \vspace{0mm}\\\vspace{0mm}%
                                                          \hfill e-mail: beck.ct@gmail.com  \vspace{0mm}\\\vspace{-4.5mm}%


    %__________________________________________________________________________________________________________________
    % Research Interests
    % \section{\mysidestyle Research\\Interests}

    % Computational Complexity, Algorithms, Combinatorics \\ 


    %__________________________________________________________________________________________________________________
    % Education
    \section{\mysidestyle Education}

    \textbf{Princeton University}, Princeton, New Jersey, USA \vspace{2mm}\\\vspace{1mm}%
    \textsl{Doctor of Philosophy: Computer Science} \hfill \textbf{ 2009 -- 2014 }\vspace{-3mm}\\\vspace{-1mm}%
%    \textsl{BInfTech, BEng (Hons)} \hfill \textbf{February 2000 -- April 2005}\vspace{-3mm}\\\vspace{-1mm}%

    \textbf{California Institute of Technology}, Pasadena, California, USA \vspace{2mm}\\\vspace{1mm}%
    \textsl{Bachelors of Science with Honor: Mathematics} \hfill \textbf{ 2005 -- 2009}\\
    \textsl{Bachelors of Science with Honor: Computer Science} \hfill \textbf{ 2005 -- 2009}\vspace{-3mm}\\%\vspace{-1mm}

    %__________________________________________________________________________________________________________________
    % Honours and Awards
    \section{\mysidestyle Honors and\\Awards} 

    Wu Prize for Excellence, 2013 \vspace{1mm}\\
    {\bf Simons Award for Graduate Students in Theoretical Computer Science}, 2012-2014		    \vspace{1mm}\\
    NSF GRFP Honorable Mention, 2009								    \vspace{1mm}\\%
    The G. Wallace Ruckert '30 Fellowship, 2009 						    \vspace{1mm}\\%
    
    %__________________________________________________________________________________________________________________
    % Professional Experience
    \section{\mysidestyle Professional\\Experience}

    \textbf{Tesla Inc.}, Palo Alto, California, USA \vspace{2mm}\\\vspace{1mm}%
    \textsl{Senior Software Developer} \hfill \textbf{Oct 2017 -- Present}\\
    Autopilot Software Infrastructure.
    
    I write C++ code that runs in the car and is used by critical software components like
    Vision, Camera, Can-bus, to communicate. Many such components are organized
    in the car computer as independent linux processes, using Posix Shared memory.
    My projects and responsbilities have included:
    
    \textsl{Implementing a series of high performance ringbuffers}

    \begin{list2}
    \item{Fixed size, multiple consumer, single producer, lock-free, and with wait-free guarantee for readers}
    \item{Fixed size, multiple producer, single consumer, lock-free, and with wait-free guarantee for reader}
    \item{Variable length single producer, single consumer ringbuffer with wait-free guarantee for all}
    \end{list2}
    
    Each of these ring buffer styles is used at different junctures in the autopilot stack. For example (1) to broad cast values read
    from a sensor to multiple parties who can read the most recent value as well as historical values in parallel,
    (2) To log timestamps associated to access events of shared memory variables (3) To hold text logs, http transactions, compressed
    binary data.
    
    \textsl{Implementing a new interface and back end for text logging in autopilot}
    
    I was assigned to replace our legacy text logging infrastructure based on the google ``glog'' library and
    to make a new real-time-friendly text logging implementation per our coding guidelines.
    It was found that this meant requiring replacing the entire glog interface as well as the implementation.
    
    I implemented a drop-in replacement for the legacy interface, using modern techniques to make it as efficient as possible.

    \textsl{Implementing efficient and generic data compression and recovery routines for real-time data values exchanged by autopilot components}

    Shared memory is used to exchange data structures ranging from a few bytes to a few megabytes. To log such data, we wanted to be able
    to generically delta-compress the changes. Historically, these data were streamed to zlib (the standard C library for lz4 compression),
    in the critical path of real-time tasks, and then handed off to a background thread to write to a pipe. We replaced all of this with variable-length shared memory ringbuffers, and implemented an efficient delta compresion
    strategy based on scanning in one pass over the old value and the new value to compute a useful diff. Our implementation made use of compiler instrincs
    to ensure that the compiler was able to use SSE instructions. Much care was taken to ensure that we do this wihtout
    breaking the strict aliasing rule.

\newpage
    
    \textsl{Implementing modern C++ utility libraries}
    
    Because many autopilot tasks have real-time critical safety requirements, many language features like exceptions
    and dynamic memory cannot be used at all while the car is moving.
    implementations of things like ``std::variant'', ``std::function'', ``std::ostream'' that provide equivalent functionality while being usable in a real-time context.

    \textsl{Contributions to our coding guidelines}

	I contribute regularly to our coding guidelines in regards to safe and modern C++.

    \textsl{Deploying ccache to speed up the autopilot build}
    
    In my first week at autopilot, I realized that we were not using ccache to accelerate incremental builds, and I believed
    that it was probably a good idea to do so. At time of initial deployment, ccache accelerated the build by a factor of more than 5 (1 hour $\rightarrow \approx$ 10 min).
    Nowadays after other improvements to the build, ccache still provides about a 2x improvement.

%    \textbf{Princeton University}, Princeton, New Jersey, USA \vspace{2mm}\\\vspace{1mm}%
%    \textsl{Research Assistant} \hfill \textbf{June 2011 -- Sept 2011}\\
%    Investigated time space tradeoffs for proof systems and other models, concluding our work on [BBI'12]. Made a conjecture implying the main result of [BIL'12] which we later proved.
%    \textsl{Teaching Assistant} \hfill \textbf{Sept 2011 -- May 2012}\\
%    Teaching assistant for undergraduate courses in theory of computation, computational geometry.
%    \textsl{Research Assistant} \hfill \textbf{June 2010 -- Sept 2010}\\
%    Investigated the classical complexity of several group-theoretic problems conjectured to have quantum speedups.
%    In an unpublished manuscript, gave evidence that this is unlikely to be the case.

%\newpage

    %\textsl{SURF Research Fellow} \hfill \textbf{June 2008 -- September 2008}\\
    %Investigated Polynomial Calculus proofs of Graph Nonisomorphism, and related issues in algebraic graph theory.
    %Mentored by Professor Richard Wilson.
       

%\newpage

    %__________________________________________________________________________________________________________________
    % Computer Skills
    \section{\mysidestyle Programming Skills} 

    C++, Python, Lua, Matlab, Bash scripting, Git. Currently learning Rust.

    \section{\mysidestyle Open Source\\Contributions}

    \textbf{visit\_struct} (Lead developer) \hfill https://github.com/cbeck88/visit\_struct \\
    A tiny library that provides for struct-field reflection in C++11. It is portable to many versions of gcc, clang, and msvc,
    and many github users tell me they have used it in production. As of 2018 it is used in Tesla autopilot code for certain code-gen tasks.

    \textbf{strict\_variant} (Lead developer) \hfill https://github.com/cbeck88/strict\_variant \\
    A simple and efficient type-safe union for C++11, which is embedded / real-time friendly, meaning it is very easy
    to use it in a way that avoids C++ exceptions and dynamic allocations. As of 2018 it is used in Tesla autopilot code.

    \textbf{spirit\_po} (Lead developer) \hfill https://github.com/cbeck88/spirit\_po \\
    A library that parses the gettext po format, and can replace the use of GNU libintl in projects that use the GNU
    gettext system for internationalization of software.
    spirit\_po is written using the boost::spirit parser framework, it is in total about 900 lines of code.
    It has been used in the Battle for Wesnoth project for several years now.

    %__________________________________________________________________________________________________________________
    % Publications

\section{\mysidestyle Academic Career}

    \textbf{Institute for Advanced Study}, Princeton, New Jersey, USA \vspace{2mm}\\\vspace{1mm}%
    \textsl{Postdoctoral Scholar} \hfill \textbf{Sept 2014 -- August 2016}\\
    Research in computational complexity theory, especially time space tradeoffs and pseudorandomness.

    \section{\mysidestyle Selected Publications}

    P. Beame, C. Beck, R. Impagliazzo. Time-Space Tradeoffs in Resolution: Superpolynomial Lower Bounds for Superlinear Space. \textsl{Proceedings of the 44th Annual ACM Symposium on Theory of Computing } (STOC 2012). {\bf Also in special issue of SIAM Journal on Computing 2016 45:4, 1612-1645 .}

\vspace{-2mm}
    C. Beck, R. Impagliazzo, S. Lovett. Large Deviation Bounds for Decision Trees and Sampling Lower Bounds for AC0-circuits. \textsl{Proceedings of the 53rd Annual IEEE Symposium on Foundations of Computer Science} (FOCS 2012). 

\vspace{-2mm}
    C. Beck, J. Nordstr{\"o}m, B. Tang. Some Tradeoffs in Polynomial Calculus. 
    \textsl{Proceedings of the 45th Annual ACM Symposium on Theory of Computing } (STOC 2013).
    

\vspace{-2mm}
    C. Beck, R. Impagliazzo. Strong ETH Holds for Regular Resolution.
    \textsl{Proceedings of the 45th Annual ACM Symposium on Theory of Computing } (STOC 2013).


    %__________________________________________________________________________________________________________________
    % Referees
%    \section{\mysidestyle Referees} 
%    {\sl Available on request.}



%______________________________________________________________________________________________________________________
% \newpage
% \section{\mysidestyle References} 
% \emph{Additional contact information available upon request.}

% \begin{tabular}{@{}p{6cm}p{6cm}}
% \textbf{Professor Sanjeev Arora}        &  \textbf{Professor Russell Impagliazzo}   \\
% Professor of Computer Science           &  Professor of Computer Science            \\
% Princeton University                    &  University of California, San Diego      \\
% Princeton, New Jersey, USA              &  San Diego, California, USA               \\
%phone: \textsl{available on request}    &  phone: \textsl{available on request}     \\
% e-mail: arora@cs.princeton.edu        &  e-mail: russell@cs.ucsd.edu    \\
% \end{tabular}

% \begin{tabular}{@{}p{6cm}p{6cm}}
% \textbf{Professor Shachar Lovett}    &  \textbf{Professor Paul Beame}        \\
% Asst. Professor of Computer Science   &  Professor of Computer Science                 \\
% University of California, San Diego       &  University of Washington       \\
% San Diego, California, USA                           &  Seattle, Washington, USA                \\
%phone: \textsl{available on request}    &  phone: \textsl{available on request}     \\
% e-mail: slovett@math.ias.edu           &  e-mail: beame@cs.washington.edu    \\
% \end{tabular}



%______________________________________________________________________________________________________________________
\end{resume}
\end{document}


%______________________________________________________________________________________________________________________
% EOF

