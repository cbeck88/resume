%______________________________________________________________________________________________________________________
% @brief    LaTeX2e Resume for Kamil K Wojcicki
\documentclass[margin,line]{resume}


%______________________________________________________________________________________________________________________
\begin{document}
\name{\Large Chris Beck}
\begin{resume}

    %__________________________________________________________________________________________________________________
    % Contact Information
    \section{\mysidestyle Contact\\Information}

                                                   %      \hfill office: +1 (609) 258 1641          \vspace{0mm}\\\vspace{0mm}%
                                                          \hfill mobile: (upon request)          \vspace{0mm}\\\vspace{0mm}%
                                                          \hfill e-mail: beck.ct@gmail.com  \vspace{0mm}\\\vspace{-4.5mm}%

    %__________________________________________________________________________________________________________________
    \section{\mysidestyle Mission}

    I enjoy thinking about complex problems and solving them from first principles, especially when performance and correctness are both critical.

    I was trained as a Theoretical Computer Scientist, with a focus in computational complexity theory.\\
    I now work as a research engineer and a cryptographer, with a strong background in high-performance low-level systems programming.
    I have expert level knowledge in Private Information Retrieval and Oblivious RAM, and familiarity with a few ZK-proving systems.

    I like working with smart, ambitious people who have clear goals and want to work on things that will impact the world positively.

    %__________________________________________________________________________________________________________________
    \section{\mysidestyle Research\\Interests}

    Computational Complexity, Pseudorandomness, Cryptography, Algorithms, Combinatorics \\

    %__________________________________________________________________________________________________________________
    % Education
    \section{\mysidestyle Education}

    \textbf{Princeton University}, Princeton, New Jersey, USA \vspace{2mm}\\\vspace{1mm}%
    \textsl{Doctor of Philosophy: Computer Science} \hfill \textbf{ 2009 -- 2014 }\vspace{-3mm}\\\vspace{-1mm}%

    \textbf{California Institute of Technology}, Pasadena, California, USA \vspace{2mm}\\\vspace{1mm}%
    \textsl{Bachelors of Science with Honor: Mathematics} \hfill \textbf{ 2005 -- 2009}\\
    \textsl{Bachelors of Science with Honor: Computer Science} \hfill \textbf{ 2005 -- 2009}\vspace{-3mm}\\%\vspace{-1mm}

    %__________________________________________________________________________________________________________________
    % Honours and Awards
    \section{\mysidestyle Honors and\\Awards} 

    Wu Prize for Excellence, 2013 \vspace{1mm}\\
    {\bf Simons Award for Graduate Students in Theoretical Computer Science}, 2012-2014		    \vspace{1mm}\\
    NSF GRFP Honorable Mention, 2009								    \vspace{1mm}\\%
    The G. Wallace Ruckert '30 Fellowship, 2009 						    \vspace{1mm}\\%

    %__________________________________________________________________________________________________________________
    % Professional Experience
    \section{\mysidestyle Professional\\Experience}

    \textbf{MobileCoin Inc.}, San Francisco, California, USA \vspace{2mm}\\\vspace{1mm}%
    \textsl{Research Engineer, Cryptographic Engineer} \hfill \textbf{Nov 2018 -- present}\\
    MobileCoin Core Engineering, Architect of MobileCoin Fog, Cryptogaphy Engineering Manager

    I was the fifth engineer hired at MobileCoin. Over my time here, my responsibilities have often shifted according to the needs of the project.
    I have often taken responsibility for mission critical tasks.
    Some highlights, in roughly reverse chronological order, include:

    \textsl{Privacy-preserving atomic swap protocol}
    \begin{list2}
    \item{MobileCoin wanted to have a privacy-preserving DEX on our chain, however, creating private smart contracts
    is a difficult research problem and the time horizon exceeded what was relevant from a business perspective. (We tabled this discussion in Q2 2021).}
    \item{In Q2 2022 I conceived of a new approach. I designed and implemented a privacy-preserving atomic swap protocol which works on our chain without creating
    smart contracts. This implementation took a few weeks. This is a peer-to-peer protocol on-top of our L1 chain. It adds no appreciable delay to a normal MobileCoin transaction.}
    \item{The initial version was specified in MCIP 31. We extended the initial design to support partial-fill trades, in MCIP 42.}
    \item{We delivered an implementation of the nodes of this peer-to-peer network at the end of Q1 2023. (https://github.com/mobilecoinofficial/deqs)}
    \end{list2}

\newpage

    \textsl{MobileCoin Fog}
    \begin{list2}
    \item{In 2018, MobileCoin faced an existential problem around how we can integrate with Signal messenger app whilst meeting Signal's privacy requirements in a scalable way.}
    \item{Leadership imagined that it would work like Monero -- the phones would download and view-key scan the entire blockchain. However, Moxie Marlinspike, the CEO of Signal, rejected this.}
    \item{They then proposed pushing this work onto the server -- SGX enclaves would view-key scan on behalf of all of the users. However, this doesn't scale in an affordable way to millions of users.}
    \item{Some engineers proposed ``sharding'' the users, to enable a trade-off between privacy and efficiency. Moxie rejected all forms of this.}
    \item{I lead a research effort to find a way that the users can find their transactions, without revealing any information to Signal or creating massive computational work on server or client side.}
    \item{I conducted an overview of the state of the art in Private Information Retrieval, and determined and Oblivious RAM inside of SGX enclaves may be a viable path forwards.}
    \item{I proposed an architecture for the MobileCoin Fog service based on experimental Oblivious RAM implementations, and provided order of magnitude estimates as to the scaling needs and costs.}
    \item{I wrote a 40 page document explaining this proposal for Signal, which they accepted.}
    \item{I implemented a version of ECIES public key encryption based on the Ristretto group, used for sending encrypted messages to the enclave.}
    \item{I proposed and implemented modifications to the MobileCoin public address format, transaction format and transaction builder to be compatible with this new system.}
    \item{I proposed and implemented a ``key-exchange random number generator'' used to privately index each user's transactions in a way that only they can detect.}
    \item{I proposed and implemented a service which loads all the transaction data into an Oblivious RAM and makes it searchable by the random values generated thusly.}
    \item{This novel system was patented and a journal publication is forthcoming.}
    \item{I developed this system and worked to support demos of our technology.}
    \item{This system has been in production for 2.5 years, having launched to production supporting millions of Signal users worldwide in February 2021, without significant incident.}
    \end{list2}

    \textsl{MobileCoin Consensus Network}
    \begin{list2}
    \item{I played a significant role in designing MobileCoin's consensus validator enclave.}
    \item{Implementation of privacy preserving proofs of membership for transaction validation.}
    \item{Developed low-level SGX infrastructure (allocator, mutex, etc.) which we use instead of the rust standard library, to reduce attack surface area. (The normal rust standard library cannot be used because system calls to the OS undermine the SGX security model.)}
    \item{Developed a structured hashing scheme compatible with protobuf-style Schema Evolution. This is used for all hashing and digital signatures within MobileCoin, and for the blockchain itself.}
    \item{Helped to improve the performance of locking schemes used in the consensus nodes.}
    \item{In later years, wrote numerous MCIPs and designed numerous features for MobileCoin, including encrypted memos, confidential token ids, minting and burning protocols.}
    \end{list2}

    \textbf{Tesla Inc.}, Palo Alto, California, USA \vspace{2mm}\\\vspace{1mm}%
    \textsl{Senior Software Engineer} \hfill \textbf{Oct 2017 -- Nov 2018}\\
    Autopilot Software Infrastructure.
    
    I wrote C++ code that runs in the car and is used by critical software components like
    Vision, Camera, Can-bus, to communicate. Many such components are organized
    in the car computer as independent linux processes, using Posix shared memory.

    \textsl{High-performance lock-free shared memory ringbuffers}
    \begin{list2}
    \item{Several implementations were given to support tradeoffs: (strong wait-free guarantee, multiple producers or consumers, fixed or variable-length payloads)}
    \item{These have various applications in the car for sharing critical data between processes, or logging activities of processes / changing of critical data}
    \item{Designed to be used in shared memory for interprocess communication, initialized lazily / on-demand, portable but optimized for x86 and arm architectures}
    \item{Implemented a clean ``single-step'' deterministic test framework for fuzzing these}
    \end{list2}

\newpage

    \textsl{Implementing a new interface and back-end for glog-style text logging}
    \begin{list2}
    \item Google's logging library was found to be making dynamic memory allocations and writing to pipes, not real-time friendly
    according to our guidelines. I was assigned to fix this.
    \item I replaced the entire glog interface and back-end, which writes over shared memory ring buffers in order not to block the critical
    path in real-time tasks.
    \item Our interface avoided several inefficiencies e.g. when several things are streamed at once into logging, we compute in advance before
    formatting how much space is needed to log everything, then we make a checkout in ringbuffer and format there without copying.
    \item We used expression templates rather than conventional ostream API to ensure that we can see all the arguments without copying any of them,
    before having to handle any one of them.
    \item This resolved frequent TOI interventions due to watchdog timeouts in developer builds
    \end{list2}

    \textsl{Implementing modern (real-time / embedded-friendly) C++ utility libraries} \\
    Because many autopilot tasks have real-time critical safety requirements, many language features like exceptions
    and dynamic memory cannot be used at all while the car is moving. I implemented equivalent versions of many STL
    classes that can be used safely in real-time tasks.
    \begin{list2}
    \item \textrm{std::variant}, \textrm{std::function}, \textrm{std::ostream}
    \item Constexpr perfect hash maps
    \end{list2}

%    \textsl{Implementing efficient and generic data compression and recovery routines for real-time data values}
%    \begin{list2}
%    \item{Shared memory is used to exchange data structures ranging from a few bytes to a few megabytes. Logging this requires compression, in the car.}
%    \item{Historically, these data were compressed using lz4
%    in the critical path of real-time tasks, and then handed off to a background thread to write to a pipe.}
%    \item{We replaced all of this with variable-length shared memory ringbuffers, and implemented an efficient delta compresion
%    strategy based on scanning in one pass over the old value and the new value to compute a useful diff.}
%    \item{Our implementation made use of compiler instrincs
%    to ensure that the compiler was able to use SSE instructions. Much care was taken to ensure that we do this wihtout
%    breaking the strict aliasing rule.}
%    \end{list2}
    
%    \textsl{Deploying ccache to speed up the autopilot build}
%    \begin{list2}
%    \item{Noticed when I was hired that ccache was not being used}
%    \item{I deployed it and showed that it sped up incremental builds by 5x (1 hour to about 12 minutes), and demonstrated how to write a simple test validating that it is configured correctly.}
%    \item{ccache is now used in local developer builds and in the build farm for all builds}
%    \end{list2}

%    \textbf{Princeton University}, Princeton, New Jersey, USA \vspace{2mm}\\\vspace{1mm}%
%    \textsl{Research Assistant} \hfill \textbf{June 2011 -- Sept 2011}\\
%    Investigated time space tradeoffs for proof systems and other models, concluding our work on [BBI'12]. Made a conjecture implying the main result of [BIL'12] which we later proved.
%    \textsl{Teaching Assistant} \hfill \textbf{Sept 2011 -- May 2012}\\
%    Teaching assistant for undergraduate courses in theory of computation, computational geometry.
%    \textsl{Research Assistant} \hfill \textbf{June 2010 -- Sept 2010}\\
%    Investigated the classical complexity of several group-theoretic problems conjectured to have quantum speedups.
%    In an unpublished manuscript, gave evidence that this is unlikely to be the case.

%\newpage

    %\textsl{SURF Research Fellow} \hfill \textbf{June 2008 -- September 2008}\\
    %Investigated Polynomial Calculus proofs of Graph Nonisomorphism, and related issues in algebraic graph theory.
    %Mentored by Professor Richard Wilson.
       

%\newpage

    %__________________________________________________________________________________________________________________
    \section{\mysidestyle Programming Proficiencies} 

    Languages: Rust, C++, C, Python, Bash scripting, Lua, Matlab
    Web Technologies: HTTP, GRPC, TLS, SQL, LMDB, Posix shared memory, Intel SGX

    \section{\mysidestyle Open Source\\Contributions}

    \textbf{visit\_struct} (Lead developer) \hfill https://github.com/cbeck88/visit\_struct \\
    A tiny library that provides for struct-field reflection in C++11. It is portable to many versions of gcc, clang, and msvc,
    and many github users tell me they have used it in production. As of 2018 it is used in Tesla autopilot code for certain code-gen tasks.

    \textbf{strict\_variant} (Lead developer) \hfill https://github.com/cbeck88/strict\_variant \\
    A simple and efficient type-safe union for C++11, which is embedded / real-time friendly, meaning it is very easy
    to use it in a way that avoids C++ exceptions and dynamic allocations. As of 2018 it is used in Tesla autopilot code.

    \textbf{spirit\_po} (Lead developer) \hfill https://github.com/cbeck88/spirit\_po \\
    A library that parses the gettext po format, and can replace the use of GNU libintl in projects that use the GNU
    gettext system for internationalization of software.
    spirit\_po is written using the boost::spirit parser framework, it is in total about 900 lines of code.
    It has been used in the Battle for Wesnoth project for several years now.

    %__________________________________________________________________________________________________________________
    % Publications

\section{\mysidestyle Academic Career}

    \textbf{Institute for Advanced Study}, Princeton, New Jersey, USA \vspace{2mm}\\\vspace{1mm}%
    \textsl{Postdoctoral Scholar} \hfill \textbf{Sept 2014 -- August 2016}\\
    Research in computational complexity theory, especially time space tradeoffs and pseudorandomness.

    \section{\mysidestyle Selected Publications}

    P. Beame, C. Beck, R. Impagliazzo. Time-Space Tradeoffs in Resolution: Superpolynomial Lower Bounds for Superlinear Space. \textsl{Proceedings of the 44th Annual ACM Symposium on Theory of Computing } (STOC 2012). {\bf Also in special issue of SIAM Journal on Computing 2016 45:4, 1612-1645 .}

\vspace{-2mm}
    C. Beck, R. Impagliazzo, S. Lovett. Large Deviation Bounds for Decision Trees and Sampling Lower Bounds for AC0-circuits. \textsl{Proceedings of the 53rd Annual IEEE Symposium on Foundations of Computer Science} (FOCS 2012). 

\vspace{-2mm}
    C. Beck, J. Nordstr{\"o}m, B. Tang. Some Tradeoffs in Polynomial Calculus. 
    \textsl{Proceedings of the 45th Annual ACM Symposium on Theory of Computing } (STOC 2013).

\vspace{-2mm}
    C. Beck, R. Impagliazzo. Strong ETH Holds for Regular Resolution.
    \textsl{Proceedings of the 45th Annual ACM Symposium on Theory of Computing } (STOC 2013).

    %__________________________________________________________________________________________________________________
    % Referees
%    \section{\mysidestyle Referees} 
%    {\sl Available on request.}



%______________________________________________________________________________________________________________________
\newpage
\section{\mysidestyle Academic References} 
\emph{Additional contact information available upon request.}

\begin{tabular}{@{}p{6cm}p{6cm}}
\textbf{Professor Sanjeev Arora}        &  \textbf{Professor Russell Impagliazzo}   \\
Professor of Computer Science           &  Professor of Computer Science            \\
Princeton University                    &  University of California, San Diego      \\
Princeton, New Jersey, USA              &  San Diego, California, USA               \\
phone: \textsl{available on request}    &  phone: \textsl{available on request}     \\
e-mail: arora@cs.princeton.edu        &  e-mail: russell@cs.ucsd.edu    \\
\end{tabular}

\begin{tabular}{@{}p{6cm}p{6cm}}
\textbf{Professor Shachar Lovett}    &  \textbf{Professor Paul Beame}        \\
Asst. Professor of Computer Science   &  Professor of Computer Science                 \\
University of California, San Diego       &  University of Washington       \\
San Diego, California, USA                           &  Seattle, Washington, USA                \\
phone: \textsl{available on request}    &  phone: \textsl{available on request}     \\
e-mail: slovett@math.ias.edu           &  e-mail: beame@cs.washington.edu    \\
\end{tabular}



%______________________________________________________________________________________________________________________
\end{resume}
\end{document}


%______________________________________________________________________________________________________________________
% EOF

